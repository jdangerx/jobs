\newenvironment{tightlist}{
\vspace{-8pt}
\begin{itemize}
  \setlength{\itemsep}{1pt}
  \setlength{\parskip}{0pt}
  \setlength{\parsep}{0pt}
  }
{\end{itemize}}

\documentclass[11pt]{article}
\usepackage[margin=1in]{geometry}
\usepackage{setspace}
\singlespace
\begin{document}
\begin{center}
  \textbf{DAZHONG JOHN XIA}\\
  1660 Madison Ave 8G\\
  New York, NY 10029\\
  (650)283-6856 \\
  djx@uchicago.edu
\end{center}

\noindent{\textbf{EXPERIENCE}} \\

\noindent\textbf{Recurse Center}\\
455 Broadway, 2nd Floor\\
New York, NY   10013 United States\\
\textbf{08/2015 - 11/2015}\\
\textbf{Hours per week:} 40\\
\textbf{Participant}\\
\textbf{Duties, Accomplishments and Related Skills:}\\
Recurse Center is a 3-month-long "programmer's retreat", where I had the opportunity to learn about any programming topics I was interested in. I spent most of my time learning Haskell, through both courses and projects.
\begin{tightlist}

\item Completed a course on Haskell fundamentals

\item Completed a course which included designing a CPU with HDL and writing an assembler, stack VM, and compiler in Haskell

\item Completed several small projects in Haskell, including a compound Chinese character generator with a browser front-end

\item Reimplemented core Git features (add, commit, and checkout) using Haskell

\end{tightlist}

\noindent\textbf{Center For Data Intensive Science at University Of Chicago}\\
900 East 57th Street\\
Rm 10148\\
Chicago, IL   60637 United States\\
\textbf{04/2015 - 08/2015}\\
\textbf{Salary:} 30.00  USD Per Hour\\
\textbf{Hours per week:} 35\\
\textbf{Software Developer}\\
\textbf{Duties, Accomplishments and Related Skills:}\\
As a software developer I wrote tools to help scientists process large amounts of data.
\begin{tightlist}

\item Main developer for Project Matsu, a collaboration between NASA and the Open Cloud Consortium to develop open source tools for cloud-based processing of satellite imagery

\item Eased geospatial data processing by writing a Python framework to run analytic scripts with Hadoop Streaming and PySpark

\item Maintained single-node installations of Hadoop and Spark on virtual machines for development

\item Shared analytic results by integrating above framework with Django, PostGIS, and GeoServer
\end{tightlist}
\textbf{Supervisor:} Maria Patterson (812-345-2654)\\
\textbf{Okay to contact this Supervisor:} Yes\\

\noindent\textbf{Heritage Chinese Center}\\
475 N Whisman Rd\\
Mountain View, CA   94043 United States\\
\textbf{06/2014 - 04/2015}\\
\textbf{Salary:} 60,000.00  USD Per Year\\
\textbf{Hours per week:} 40\\
\textbf{Software Developer}\\
\textbf{Duties, Accomplishments and Related Skills:}\\
As a software developer I Worked on a two-person team to build a Django-based
webapp to supplement a Chinese education curriculum.
\begin{tightlist}
\item Brainstormed and tweaked game and website designs with teachers

\item Used Phaser, a JavaScript game framework, to make vocabulary-building webgames

\item Prototyped in-browser Chinese linguistic tone recognition with JavaScript and Web Audio API

\item Installed and administered PostgreSQL database on AWS server

\item Ported webapp to iOS using Phonegap
\end{tightlist}
\textbf{Supervisor:} Sushu Xia (650-283-7008)\\
\textbf{Okay to contact this Supervisor:} Yes\\

\noindent\textbf{Computational Nuclear Engineering Research Group - University of Wisconsin-Madison}\\
1500 Engineering Dr\\
Madison, WI   53706 United States\\
\textbf{10/2014 - 03/2015}\\
\textbf{Salary:} 25,000.00  USD Per Year\\
\textbf{Hours per week:} 20\\
\textbf{Associate Research Specialist}\\
\textbf{Duties, Accomplishments and Related Skills:}\\
At CNERG, I continued working with and expanding the nuclear engineering tool suite I had been working on at the University of Chicago.
\begin{tightlist}

\item Rebuilt xsgen, an open-source nuclear cross-section library generator written in Python

\item Co-evolved particle transport and transmutation to generate more accurate cross-section libraries

\item Built plug-ins to integrate OpenMC, an open-source Monte Carlo particle transport code, and ORIGEN, a transmutation code, as computation backends

\end{tightlist}
\textbf{Supervisor:} Anthony Scopatz (512-827-8239)\\
\textbf{Okay to contact this Supervisor:} Yes\\

\noindent\textbf{Flash Center for Computational Science - University of Chicago}\\
5747 S. Ellis Ave, 3rd floor\\
Chicago, IL   60637 United States\\
\textbf{10/2012 - 06/2014}\\
\textbf{Hours per week:} 8\\
\textbf{Research Assistant}\\
\textbf{Duties, Accomplishments and Related Skills:}\\
As a research assistant I mainly contributed to PyNE, an open-source nuclear
engineering toolkit written mostly in Python.
\begin{tightlist}

\item Imported PNNL materials compendium into PyNE database

\item Constructed infrastructure for accessing and processing ENDF nuclear data using NumPy

\item Optimized data parsing for speed using Cython

\item Measured and visualized software performance using matplotlib

\end{tightlist}
\textbf{Supervisor:} Anthony Scopatz (512-827-8239)\\
\textbf{Okay to contact this Supervisor:} Yes\\

\noindent\textbf{Mission Street Manufacturing}\\
807 Mission St\\
Santa Barbara, CA   93101 United States\\
\textbf{06/2013 - 08/2013}\\
\textbf{Hours per week:} 40\\
\textbf{Software Intern}\\
\textbf{Duties, Accomplishments and Related Skills:}\\
As a software intern I worked with three other people to build software to simplify the 3D printing process.

\begin{tightlist}

\item Enabled children to design and print 3D models by connecting Raspberry Pi, 3D printer, iPads, and AWS server

\item Coordinated server-side 3D model processing with Python and Redis

\item Wrote native iOS app in Objective-C to design 3D objects by revolving a 2D sketch

\item Wrote server-side tool to extrude bitmap images into 3D models with Python and Numpy

\item Prototyped constructive solid geometry rendering engine for iOS using Objective-C\\

\end{tightlist}

\noindent\textbf{EDUCATION} \\
\textbf{University of Chicago}, Chicago, IL \\
\textbf{BA: Physics with Honors}, June 2014 \\
\textbf{GPA:} 3.56/4.00 \\\\

%% \textbf{Exploratorium}, San Francisco, CA \\
%% \textbf{Tinkering Studio Volunteer}, June 2011 - present
%% \begin{tightlist}
%%   \item Design and manufacture lasercut wood and plastic parts with SolidWorks
%%   \item Participate in design, testing, building, and maintenance of Tinkering Studio activities
%% \end{tightlist}
%% \textbf{Kasevich Group}, Stanford, CA \\
%% \textbf{Research Assistant}, June 2012 - September 2012
%% \begin{tightlist}
%%   \item Prototyped, designed, and implemented PCB laser frequency locking system
%%   \item Built a set of heaters to increase fiber modulator efficiency
%% \end{tightlist}
%% \textbf{LEADERSHIP} \\
%% \textbf{Big Projects}, Max Palevsky Scavenger Hunt, University of Chicago, May 2012 - June 2014\\
%% \vspace{-14pt}
%% \begin{tightlist}
%%   \item Design whimsical creations such as drink-mixing piano and
%%   mechanized campus map
%%   \item Organize and build multiple projects in 4 days with extremely
%%   limited materials
%% \end{tightlist}
%% \textbf{Robotics Team Captain}, Palo Alto High School, Palo Alto, CA,
%% August 2009 - June 2010
%% \begin{tightlist}
%%   \item Organized and trained a team of students to compete in the \emph{FIRST}
%%     Robotics Challenge
%%   \item Collaborated with other students to design and build successful robot in six weeks
%% \end{tightlist}
\noindent\textbf{SKILLS} \\
%% things that I type into a text editor:
\textbf{Programming Languages:} Proficient with Python, Haskell, JavaScript, HTML5/CSS, \LaTeX\\
%% things that I type into a command prompt:
\textbf{Tools:} Linux, Git, Django, Flask, Hadoop, Spark, PostgreSQL, PostGIS, AWS \\
\end{document}
